%!TEX root = labreport.tex
%%%%%%%%%%%%%%%%%%%%%%%%%%%%%%%%%%%%%%%%%%%%%%%%%%%%%%%%%%%%%%%%%%%%%%%%%%%

\documentclass[sigconf]{acmart}

\usepackage{booktabs} % For formal tables
\usepackage{amsmath}

% Copyright
\setcopyright{none}
%\setcopyright{acmcopyright}
%\setcopyright{acmlicensed}
%\setcopyright{rightsretained}
%\setcopyright{usgov}
%\setcopyright{usgovmixed}
%\setcopyright{cagov}
%\setcopyright{cagovmixed}


% DOI
% \acmDOI{10.475/123_4}

% ISBN
% \acmISBN{123-4567-24-567/08/06}

%Conference
\acmConference[SEEMOO PhySec (Lab RDS) 2017/18]{Physical Layer Security
in Wireless Systems (Lab RDS) 2017/18}{February 2018}{Darmstadt, Germany}

\acmYear{2018}
\copyrightyear{2018}

% \acmArticle{4}
% \acmPrice{15.00}

\settopmatter{printacmref=false}

\begin{document}
\title{Reactive Jamming}

% Comment out the following for you final document
\subtitle{Lab Report}



\author{Daniel May}
\author{Simon Schmitt}
\affiliation{%
  \institution{Technische Universtit\"at Darmstadt}
}
\email{daniel\_nicolas.may@stud.tu-darmstadt.de}
\email{simon\_johannes.schmitt@stud.tu-darmstadt.de}

% The default list of authors is too long for headers.
\renewcommand{\shortauthors}{Daniel May \& Simon Schmitt}


\begin{abstract}
%  Answer the following questions with roughly one sentence each:
%
%  One-page summary:
%  \begin{itemize}
%  \item What is the topic of your main seminar paper?
%  \item What problem does it solve?
%  \item Why is that topic/problem important?
%  \item What methodologies do the authors apply?
%  \item What are the main contributions of the paper?
%  \item What are the key findings/results of the paper?
%  \end{itemize}
%  
%  Final seminar paper:
%  \begin{itemize}
%  \item What is your research question?
%  \item Why are that question and your topic important?
%  \item How did you proceed to answer the question?
%  \item What (do you think) is the answer to your question?
%  \item Give an overall opinion on your topic.
%  \item If you have results, describe them.
%  \item What is the impact of the answers to your questions?
%  \end{itemize}
This lab report presents the creation of an reactive jammer. We will describe how the frame handling
works on the WARP and how it can be used to suppress individual targeted devices or communications
respectively. At the end of this report we evaluate the performance of our jammer and discuss
possible improvements.


\end{abstract}

\maketitle

\section{Introduction}
Wireless signals, as they are used in most of today's analog or digital communications, are very
sensitive and affectable by the environment. Signals with the same frequency can interfere and
suppress each other. This effect is typically used by jammers to prevent a certain receiver from
decoding a signal. While jamming is typically associated with malicious behaviour or within military
conflicts to hinder an opposing party from exchanging information, there also exists other jamming
schemes, so called friendly jamming. Friendly jamming can be used to protect vulnerable systems from
adversarial actions, e.g., pacemakers that can be wirelessly reprogrammed. More recent work also
demonstrated that secrete key-exchanges can be realized at the physical layer utilizing a jammer.

The objective of this lab was to create a reactive WiFi jammer using the Wireless Open-Access
Research Platform (WARP). WARP is a programmable Software-Defined Radio (SDR) which provides a basic
implementation of the 802.11g WiFi standard. The architecture of the WARP allows to transmit frames
while still receiving a signal. Thus WiFi transmissions with a certain Medium Access Control (MAC)
address can be analyzed and jammed if they are matching a target address.

In comparison with existing jammers this approach is more precise as it only suppresses the signals
of a certain target, while still allowing the communication of other devices. This also results in
much lower power-consumptions, due to the smaller amount of frames that have to be jammed.

%Write a short paragraph (5-15 lines) on each of the following tasks:
%\begin{itemize}
%\item Motivate your topic in general.
%\item Why is your research question important in that field?
%\item Give one practical example.
%\item To what existing work is your topic related, what has been done there?
%\item What are the (planned) main contributions of your paper? e.g., a
%new attacker model, a summary, a comparison, \dots
%\item Give an outline of the paper: describe each of your (planned)
%sections in one sentence.
%\end{itemize}


%\section{Background and Related Work}

\section{Implementation}

\section{Conclusion and Take-Away}

\section{Future Work}


\bibliographystyle{ACM-Reference-Format}
\bibliography{library} 

\end{document}
